% -----
% COMP2550 proposal
% CHRISTOPHER CLAOUE-LONG
% -----
% -
% - DOCUMENT GEOMETRY SETUP
\documentclass[12pt,a4paper]{article}
\usepackage[margin=20mm]{geometry}
\usepackage{lastpage} % to display the page number down the bottom
\makeatletter \renewcommand{\@oddfoot}{\hfil Page \thepage\ of \pageref{LastPage} \hfil} \makeatother % Page X of Y down the bottom
% -
% - FONT
\usepackage{amsmath,amsthm,amssymb,graphicx,epstopdf,datetime,multicol,verbatim,ulem,alltt,multirow}
\DeclareGraphicsRule{.tif}{png}{.png}{`convert #1 `dirname #1`/`basename #1 .tif`.png}
\usepackage[sc]{mathpazo} % Palatino maths fonts - acceptable fonts for typesetting, also sets these fonts for use in math mode
\linespread{1.05}
\usepackage[T1]{fontenc}
\usepackage[bitstream-charter]{mathdesign}
% -
% - MISC. PACKAGES
\usepackage[usenames,dvipsnames,svgnames,table]{xcolor}
\usepackage{hyperref}
\hypersetup{
colorlinks,
citecolor=black,		% - Citation colour
filecolor=black,		% - File colour
linkcolor=black,		% - Link colour
urlcolor=black		% - URL colour
}\urlstyle{same}
% -
% -
% - MISC. SYMBOLS AND COMMANDS
\newcommand{\HUGE}[1]{\textbf{\Huge #1}}
\newcommand{\ITALIC}[1]{\textit{#1}}
\newcommand{\BOLDL}[1]{\textbf{\large #1}}
\newcommand{\BOLD}{\textbf}
\newcommand{\Hrule}{\textcolor{blue}{\rule{\linewidth}{0.5mm}}} 
% Defines a new command for the horizontal lines, change thickness here
\newcommand{\htab}{\hspace*{0.63cm}}
% -
% -----
% BEGIN DOCUMENT
% -----
\begin{document}
% -----
% - Title
{\center{\Hrule\vspace{0.2em}
	%
	\HUGE{COMP2550/COMP3130 ANU\\Main Project Proposal}\\
	\BOLDL{
		Christopher Claou\'e-Long
		(\href{mailto:u5183532@anu.edu.au}
		{\ITALIC{\underline{\smash{u5183532@anu.edu.au}}}})\\
		Jimmy Lin 
		(\href{mailto:u5223173@anu.edu.au}
		{\ITALIC{\underline{\smash{u5223173@anu.edu.au}}}})\\
	}
	\BOLD{Last typeset: \currenttime, \today}
\\\Hrule}}
% -
% -
\begin{multicols}{2}
\section{Introduction}
    problem motivation, background, application. \\
    what we are going to achieve?\\ 
    We are going to use the open-source library OpenCV and Darwin to achieve the framework of the CRF-based saliency detection. 

\section{Related Work}
    a brief introduction of the alternatives.
\section{Our Approach}

\subsection{Model Formulation}

\subsection{Feature Extraction}

\section{Possible Improvements}

% References
\begin{thebibliography}{8}
    \footnotesize
    \bibitem 1 Gould, Stephen. "DARWIN: A Framework for Machine Learning and Computer Vision Research and Development." \textit{Journal of Machine Learning Research 13 (2012): 3533-3537}. 
    \bibitem 2 Liu, Tie, et al. "Learning to detect a salient object." \textit{Pattern Analysis and Machine Intelligence, IEEE Transactions on 33.2 (2011): 353-367}. 
    \bibitem 3 Itti, Laurent, Christof Koch, and Ernst Niebur. "A model of saliency-based visual attention for rapid scene analysis."\textit{ Pattern Analysis and Machine Intelligence, IEEE Transactions on 20.11 (1998): 1254-1259}.
    \bibitem 4 Ma, Yu-Fei, and Hong-Jiang Zhang. "Contrast-based image attention analysis by using fuzzy growing."\textit{ Proceedings of the eleventh ACM international conference on Multimedia. ACM, 2003}. 
\end{thebibliography}

\end{multicols}

\vfill\Hrule
\end{document}
% -----
% END OF LINE
% -----
