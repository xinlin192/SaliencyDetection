\documentclass[9pt]{beamer}
\usepackage{./Amsterdam}
%\usetheme{Marburg}
\usetheme{Goettingen}

% package list
\usepackage{graphicx}
\usepackage{wrapfig}

% prensentation info
\title{{Application of Conditional Random Field in Image Salient Object Detection with Local, Regional, Global Feature Extraction}}
\author{Jimmy Lin \\Chris Claoue Long}
\institute{Dr. Stephen Gould\\[0.3cm] College of Engineering and Computer Science \\Australian National University}
\date{\today}

% new command definition
\DeclareMathOperator*{\argmin}{arg\,min}
\DeclareMathOperator*{\argmax}{arg\,max}


%%% beginning of document
\begin{document}
\begin{large}
\frame{\titlepage }
\end{large}

\section{Introduction}
\newpage
\section{Related Works}
\section{Formulation}
\newpage
\section{Feature Extraction}
\newpage
\subsection{Multiscale Contrast}
\newpage
\subsection{Center-Surround Histogram}
\newpage
\subsection{Color Spatial Distribution}
\newpage
\section{Conditional Random Field}
\newpage
\subsection{Learning}
\newpage
\subsection{Inference}
\newpage
\section{Reference}
\begin{thebibliography}{9}
        \bibitem{ConcreteMath} Liu, Tie, et al. "Learning to detect a salient object."\textit{ Computer Vision and Pattern Recognition, 2007. CVPR'07. IEEE Conference on. IEEE, 2007. }
    \end{thebibliography}
\end{document}
